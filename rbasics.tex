\documentclass[]{tufte-book}

% ams
\usepackage{amssymb,amsmath}

\usepackage{ifxetex,ifluatex}
\usepackage{fixltx2e} % provides \textsubscript
\ifnum 0\ifxetex 1\fi\ifluatex 1\fi=0 % if pdftex
  \usepackage[T1]{fontenc}
  \usepackage[utf8]{inputenc}
\else % if luatex or xelatex
  \makeatletter
  \@ifpackageloaded{fontspec}{}{\usepackage{fontspec}}
  \makeatother
  \defaultfontfeatures{Ligatures=TeX,Scale=MatchLowercase}
  \makeatletter
  \@ifpackageloaded{soul}{
     \renewcommand\allcapsspacing[1]{{\addfontfeature{LetterSpace=15}#1}}
     \renewcommand\smallcapsspacing[1]{{\addfontfeature{LetterSpace=10}#1}}
   }{}
  \makeatother
\fi

% graphix
\usepackage{graphicx}
\setkeys{Gin}{width=\linewidth,totalheight=\textheight,keepaspectratio}

% booktabs
\usepackage{booktabs}

% url
\usepackage{url}

% hyperref
\usepackage{hyperref}

% units.
\usepackage{units}


\setcounter{secnumdepth}{2}

% citations
\usepackage{natbib}
\bibliographystyle{apalike}

% pandoc syntax highlighting

% longtable
\usepackage{longtable,booktabs}

% multiplecol
\usepackage{multicol}

% strikeout
\usepackage[normalem]{ulem}

% morefloats
\usepackage{morefloats}


% tightlist macro required by pandoc >= 1.14
\providecommand{\tightlist}{%
  \setlength{\itemsep}{0pt}\setlength{\parskip}{0pt}}

% title / author / date
\title{Getting used to R, RStudio, and RMarkdown}
\author{Chester Ismay}
\date{2016-08-14}

\usepackage{booktabs}
\usepackage{longtable}
\usepackage{framed,color}
\definecolor{shadecolor}{RGB}{248,248,248}

\ifxetex
  \usepackage{letltxmacro}
  \setlength{\XeTeXLinkMargin}{1pt}
  \LetLtxMacro\SavedIncludeGraphics\includegraphics
  \def\includegraphics#1#{% #1 catches optional stuff (star/opt. arg.)
    \IncludeGraphicsAux{#1}%
  }%
  \newcommand*{\IncludeGraphicsAux}[2]{%
    \XeTeXLinkBox{%
      \SavedIncludeGraphics#1{#2}%
    }%
  }%
\fi

%% Need to clean up
\newenvironment{rmdblock}[1]
  {\begin{shaded*}
  \begin{itemize}
  \renewcommand{\labelitemi}{
    \raisebox{-.7\height}[0pt][0pt]{
  %    {\setkeys{Gin}{width=3em,keepaspectratio}\includegraphics{images/#1}}
    }
  }
  \item
  }
  {
  \end{itemize}
  \end{shaded*}
  }
%% Probably can be omitted
\newenvironment{rmdnote}
  {\begin{rmdblock}{note}}
  {\end{rmdblock}}
\newenvironment{rmdcaution}
  {\begin{rmdblock}{caution}}
  {\end{rmdblock}}
\newenvironment{rmdimportant}
  {\begin{rmdblock}{important}}
  {\end{rmdblock}}
\newenvironment{rmdtip}
  {\begin{rmdblock}{tip}}
  {\end{rmdblock}}
\newenvironment{rmdwarning}
  {\begin{rmdblock}{warning}}
  {\end{rmdblock}}
\newenvironment{learncheck}
  {\begin{rmdblock}{warning}}
  {\end{rmdblock}}
\newenvironment{review}
  {\begin{rmdblock}{warning}}
  {\end{rmdblock}}

% To tweak tufte layout
\geometry{
  left=0.8in, % left margin
  textwidth=35pc, % main text block
  marginparsep=1pc, % gutter between main text block and margin notes
  marginparwidth=8pc % width of margin notes
}

\begin{document}

\maketitle



{
\setcounter{tocdepth}{1}
\tableofcontents
}

\chapter{Introduction}\label{intro}

This book was written using the \textbf{bookdown} R package from Yihui
Xie. You can find different formats for the book by clicking on the save
icon \includegraphics[width=0.19in]{screenshots/save_icon} in the top
pane of this book website. HTML is the preferred format but PDF and ePub
formats are also available.

This resource is designed to provide new users to R, RStudio, and
RMarkdown with the introductory steps needed to begin their own
reproducible research. A review of many of the common R errors
encountered (and what they mean in laymen's terms) is also provided.
Many screenshots and GIFs will be included, but if further clarification
is needed on these or any other aspect of the book, please create a
GitHub issue \href{https://github.com/ismayc/rbasics/issues}{here} or
email \href{mailto:chester.ismay@gmail.com}{me} with a reference to the
error/area where more guidance is necessary.

This book will evolve and be updated as needed based on feedbackS. Check
the date at the beginning of this chapter to see when it was last
updated. In addition, each individual chapter shows the time of last
update.

\chapter{Why R?}\label{whyR}

If you are brand new to R and programming, you may be scared. You aren't
used to having to type commands to tell the computer what to do. You may
be more used to using drop-down menus and other graphical user
interfaces that allow you to pick what you'd like to do. So why are so
many companies, colleges/universities, and individuals of all
disciplinary backgrounds shifting towards using R?

There are lots of answers to this question, but some of the most
important for us now are:

\begin{enumerate}
\def\labelenumi{\arabic{enumi}.}
\item
  R is free. RStudio is free.

  One of the biggest perks about working with R and RStudio is that they
  are both provided free of charge to use. R is an open-source
  programming language that has grown tremendously in recent years with
  developers adding more functionality and packages on a daily basis.
  Where other more proprietary packages are sometimes stuck in the dark
  ages (the 1990s, for example) of development and can be incredibly
  expensive to purchase, R continues to be a free alternative that
  allows users of all levels to contribute.

  RStudio is a graphical user interface that allows one to write R code
  and view the results of that code in an easy way. It is also free to
  download and work with.
\item
  Analyses done using R are reproducible.

  As many scientific fields push towards more reproducible analyses, the
  point-and-click proprietary systems actually serve as a hindrance to
  this process. If you need to re-run your analysis using these systems,
  you'll need to carefully copy-and-paste your analysis and plots into
  your text editors from potentially beginning to end. Anyone that has
  done this sort of copy-and-pasting knows that it is prone to errors
  and incredibly tedious.

  If you use the workflows described in this book, your analyses will be
  reproducible so you don't need to worry about these copy-and-pasting
  issues. As you might have guessed by now, it would be much better to
  be able to update your code/data inputs and re-run all of your
  analysis than to have to worry about manually moving your results from
  one program to another. Reproducibility also helps you as a programmer
  since your greatest collaborator will probably be yourself a few
  months or years down the road. Instead of having to carefully write
  down all the steps you took to find the right drop-down menu option,
  your entire code is stored.
\item
  Using R makes collaboration easier.

  This also helps you with collaboration since, as you will see later,
  you can share an RMarkdown file containing all of your analysis,
  documentation, commentary, and the code to others. This reduces the
  time to needed to work with others and reduces the likelihood of
  errors being made in following along with point-and-click analyses.
  The mantra here will be to \textbf{Say No to Copy-And-Paste!} both for
  your sanity and for the sake of science.
\item
  Finding answers to questions is much simpler.

  If you have ever had an issue with software, you know how difficult it
  is to find answers to your questions. ``How can I describe the process
  to someone else? Do I need to take screenshots? Do I really need to
  call IT and wait for hours for someone to respond?'' R is a
  programming language and so it is much easier (after a bit of
  practice) to use Google or Stack Overflow to find answers to your
  questions. You'll be amazed at how many other users have encountered
  the same sorts of errors you will see when you begin.

  I frequently (almost on a daily basis) Google things like ``How do I
  make a side-by-side boxplot in R coloring by a third variable?''.
  You'll become better at working with R by reaching out to others for
  help and by answering questions that others have. In addition, Chapter
  \ref{errors} describes many common errors and how you can fix them.
\item
  Struggling through programming helps you learn.

  We all know that learning isn't easy. Do you have trouble remembering
  how to follow a list of more than 10 steps or so? Do you find yourself
  going back over and over again because you can't remember what step
  comes next in the process? This is extremely common especially if you
  haven't done the procedure in awhile. Learning via following a
  procedure is easy in the short-term, but can be extremely frustrating
  to remember in the long-term. Programming (if done well) promotes
  long-term thinking to short-term fixes.

  One unfortunate thing that we frequently take for granted is that our
  brain tricks us into picking the easy route. If you truly want to
  learn how to do something (like programming with R), you'll need to
  feel frustrated at times. Any time you learn something you've been
  frustrated. (We tend to forget all the frustration and only think
  about where we currently are.) R still frustrates me from time to
  time, but I grow through practice and I look forward to the
  challenges. Hadley Wickham encapsulated this phenomenon nicely in the
  Prologue of the book ``Hands-On Programming with R''
  \citep{handson2014}:

  \begin{quote}
  As you learn to program, you are going to get frustrated. You are
  learning a new language, and it will take time to become fluent. But
  frustration is not just natural, it's actually a positive sign that
  you should watch for. Frustration is your brain's way of being lazy;
  it's trying to get you to quit and go do something easy or fun. If you
  want to get physically fitter, you need to push your body even though
  it complains. If you want to get better at programming, you'll need to
  push your brain. Recognize when you get frustrated and see it as a
  good thing: you're now stretching yourself. Push yourself a little
  further every day, and you'll soon be a confident programmer.
  \end{quote}
\end{enumerate}

\textbf{Last updated:}

\begin{verbatim}
## [1] "By Chester on Sunday, August 14, 2016 11:43:33 PDT"
\end{verbatim}

\chapter{R and RStudio Basics}\label{rstudiobasics}

\section{What is R?}\label{what-is-r}

In Chapter \ref{whyR}, I discussed many of the reasons why you should be
doing your analyses (especially those of the data type) using R. If you
skipped over that chapter in the hopes of just hopping into learning
about R, I request that you to go back to it and carefully read it over.
As you begin working with R, it is especially important to review that
introductory chapter from time to time.

\subsection{R beginnings}\label{r-beginnings}

R was created by a group of statisticians who wanted an open-source
alternative to the costly proprietary options. Being created by
statisticians (instead of computer scientists) means that R has some
quirky aspects to it that take a little bit of time to get used to.
We'll see that many packages have been developed to help with this and
that you don't need to have advanced degrees in Statistics to be able to
work with R now.

Getting back to the development of R\ldots{} R was created by
\textbf{R}oss Ihaka and \textbf{R}obert Gentleman in New Zealand at the
University of Auckland. It is a spin-off of the S programming language
and is named partly after the first names of its developers (as you can
see in the emphasis above). The beginning ideas for creating R came in
1992 and the first version of R was released in 1994. You can find much
more about the background of R and its features as well as its
connections to the S language on its
\href{https://en.wikipedia.org/wiki/R_(programming_language)}{Wikipedia
page}.

\subsection{R packages}\label{r-packages}

I first learned to use R while a graduate student at Northern Arizona
University from \href{http://www.stat.colostate.edu/~pturk/}{Dr.~Philip
Turk} in 2007. At the time, I never thought that R could have exploded
in users as we have seen since 2011. I never would have thought that
students taking an introductory statistics course would be encouraged to
learn to use R.

In 2007, it was still largely esoteric and tricky language used by
statisticians to do analyses. Getting used to the syntax for producing
plots and working with data was especially tricky for those with little
to no programming experience. So what has changed since 2007 about
learning R?

I believe one of the biggest developments has been the creation of
packages to make R easier to work with for newbies. Packages are created
by users of R to increase the functionality of the base R installation.
Packages created by Hadley Wickham and others recently have greatly
expanded the capabilities of R, while also working to make beginning
with R simpler. From the Wikipedia page referenced earlier, as of
January 2016, there were around 7800 additional R packages available on
common R repositories.\footnote{You'll see how to download these
  packages via \texttt{install.packages("dplyr")} and load them into
  your current R working environment via \texttt{library("dplyr")}, for
  example, in Chapter \ref{rmdanal}.}

Another great development is the graphical user interface called RStudio
and the package developed by the those that work for RStudio, Inc.
called \texttt{rmarkdown}. We will discuss \texttt{rmarkdown} (also
referred to as RMarkdown) in a Chapter \ref{rmarkdown}, and will now
focus on discussing RStudio.

\section{What is RStudio?}\label{what-is-rstudio}

RStudio is a powerful, free, open-source integrated development
environment for R. It began development in 2010 and its first beta
release came in February 2011. It is available in two editions: RStudio
Desktop and RStudio Server. This book will focus mostly on using the
RStudio Server, but both versions are nearly identical to work with.

You can find instructions linked below for downloading R and RStudio on
Windows and Mac machines. If you are using RStudio Server, your
professor and members of your organization's IT department have done
these steps for you. On the RStudio Server you log on using a web
browser to an account sitting on the cloud. There are many advantages to
using the RStudio Server for the beginning user including sharing of R
projects to help with feedback and error resolution. Installation of
software can also cause its own headaches and this is eliminated by
using the RStudio Server.

After you complete a few months of work with the RStudio Server, it is
recommended that you download RStudio Desktop to your computer. The
instructions to do so are below.

\subsection{Installing R and RStudio
Desktop}\label{installing-r-and-rstudio-desktop}

A step-by-step guide to installing R and RStudio Desktop with
screenshots can be found

\begin{itemize}
\tightlist
\item
  \href{http://www.reed.edu/data-at-reed/software/R/r_studio.html}{here}
  for the Mac and
\item
  \href{http://www.reed.edu/data-at-reed/software/R/r_studio_pc.html}{here}
  for a PC.
\end{itemize}

Unless you plan to create PDF documents (which requires a multiple
gigabyte download of LaTeX) you can skip some of the later steps of the
installation. It is recommended that you select \textbf{HTML} as the
Default Output Format for RMarkdown. You'll see more about this in
Chapter \ref{rmarkdown}.

\section{Working in RStudio Server}\label{working-in-rstudio-server}

\begin{itemize}
\tightlist
\item
  RStudio Server
\item
  Screenshots of RStudio frames?
\end{itemize}

\textbf{Last updated:}

\begin{verbatim}
## [1] "By Chester on Sunday, August 14, 2016 11:43:33 PDT"
\end{verbatim}

\chapter{R Markdown}\label{rmarkdown}

\begin{itemize}
\tightlist
\item
  Walk through the components of an R Markdown file
\item
  Resource for Markdown:
  \url{https://github.com/adam-p/markdown-here/wiki/Markdown-Cheatsheet}
\item
  RMarkdown chunk options
\item
  Help -\textgreater{} Cheatsheets
\end{itemize}

\textbf{Last updated:}

\begin{verbatim}
## [1] "By Chester on Sunday, August 14, 2016 11:43:33 PDT"
\end{verbatim}

\chapter{Introductory R analysis using R Markdown}\label{rmdanal}

\begin{itemize}
\tightlist
\item
  A beginning workflow
\item
  ``File organization and naming are powerful weapons against chaos.'' -
  Jenny Bryan
\item
  Give an introduction into using R with periodic table dataset
\item
  Mean, median, standard deviation, five-number summary, distribution
\item
  Some content to cover:

  \begin{itemize}
  \tightlist
  \item
    data structures (vectors, lists, data frames, matrices)
  \item
    indexing/subsetting
  \item
    functions (default arguments)
  \item
    Case matters in R!
  \item
    Why do some arguments require quotations and others don't?
  \end{itemize}
\item
  R Markdown templates
\end{itemize}

\textbf{Last updated:}

\begin{verbatim}
## [1] "By Chester on Sunday, August 14, 2016 11:43:33 PDT"
\end{verbatim}

\chapter{Deciphering Common R Errors}\label{errors}

\url{https://github.com/noamross/zero-dependency-problems/blob/master/misc/stack-overflow-common-r-errors.md}

\url{http://blog.revolutionanalytics.com/2015/03/the-most-common-r-error-messages.html}

\textbf{Last updated:}

\begin{verbatim}
## [1] "By Chester on Sunday, August 14, 2016 11:43:33 PDT"
\end{verbatim}

\bibliography{bib/packages.bib,bib/books.bib,bib/articles.bib}



\end{document}
